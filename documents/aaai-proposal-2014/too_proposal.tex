\documentclass[12pt]{article}
\usepackage{latexsym,amssymb,amsmath} % for \Box, \mathbb, split, etc.
% \usepackage[]{showkeys} % shows label names
\usepackage{cite} % sorts citation numbers appropriately
\usepackage{path}
\usepackage{url}
\usepackage{verbatim}
\usepackage[pdftex]{graphicx}
\usepackage{color}

% horizontal margins: 1.0 + 6.5 + 1.0 = 8.5
\setlength{\oddsidemargin}{0.0in}
\setlength{\textwidth}{6.5in}
% vertical margins: 1.0 + 9.0 + 1.0 = 11.0
\setlength{\topmargin}{0.0in}
\setlength{\headheight}{12pt}
\setlength{\headsep}{13pt}
\setlength{\textheight}{625pt}
\setlength{\footskip}{24pt}

\renewcommand{\textfraction}{0.10}
\renewcommand{\topfraction}{0.85}
\renewcommand{\bottomfraction}{0.85}
\renewcommand{\floatpagefraction}{0.90}

\makeatletter
\setlength{\arraycolsep}{2\p@} % make spaces around "=" in eqnarray smaller
\makeatother

% change equation, table, figure numbers to be counted inside a section:
\numberwithin{equation}{section}
\numberwithin{table}{section}
\numberwithin{figure}{section}

% begin of personal macros
\newcommand{\half}{{\textstyle \frac{1}{2}}}
\newcommand{\eps}{\varepsilon}
\newcommand{\myth}{\vartheta}
\newcommand{\myphi}{\varphi}

\newcommand{\IN}{\mathbb{N}}
\newcommand{\IZ}{\mathbb{Z}}
\newcommand{\IQ}{\mathbb{Q}}
\newcommand{\IR}{\mathbb{R}}
\newcommand{\IC}{\mathbb{C}}
\newcommand{\Real}[1]{\mathrm{Re}\left({#1}\right)}
\newcommand{\Imag}[1]{\mathrm{Im}\left({#1}\right)}

\newcommand{\norm}[2]{\|{#1}\|_{{}_{#2}}}
\newcommand{\abs}[1]{\left|{#1}\right|}
\newcommand{\ip}[2]{\left\langle {#1}, {#2} \right\rangle}
\newcommand{\der}[2]{\frac{\partial {#1}}{\partial {#2}}}
\newcommand{\dder}[2]{\frac{\partial^2 {#1}}{\partial {#2}^2}}

\newcommand{\nn}{\mathbf{n}}
\newcommand{\xx}{\mathbf{x}}
\newcommand{\uu}{\mathbf{u}}

\newcommand{\junk}[1]{{}}

% set two lengths for the includegraphics commands used to import the plots:
\newlength{\fwtwo} \setlength{\fwtwo}{0.45\textwidth}


\renewcommand{\labelitemi}{}
\renewcommand{\labelitemii}{}
\renewcommand{\labelitemiii}{}


% end of personal macros
% \input{inputFile.tex}


\begin{document}
\DeclareGraphicsExtensions{.jpg}



\begin{center}
\textbf{\Large }AAAI Fellowship Application\\[6pt] 
%\textbf{\Large } \\[6pt]
%\textbf{\Large } \\[6pt]
John Oberlin\\
Brown University, 2014\\
\end{center}

\section{Introduction}
%\cite{FooBar}

State of the art techniques in object detection and pose estimation
are powerful and general but usually run at a rate less than 1 Hz. This makes
it difficult to employ such techniques in real-time human-computer interaction.
This document outlines a simple, robust framework for object detection which 
trades a large memory overhead for improvements in latency and total throughput 
of detections. Included is a workflow for that framework which makes training
and calibration first intuitive and then automatic.

Our overall pipeline can be described as:

\begin{enumerate}
  \item Collect RGB-D data for objects.
  \item Train BoW model and kNN classifiers for objects.
  \item Use the classifiers and additional logic to provide 3D detections pose estimates of objects.
\end{enumerate}

\section{Basic Techniques and Detector Structure}

Objectness
Fast Keypoints
SIFT descriptors
KMeans
BoW
Color Histogram
Depth Histogram
kNN
MDP

Green Boxes
Blue Boxes
Red Boxes

\section{Teaching and Employing the System}
This is teaching not training because it is an interaction used to collect
data not a script used to optimize a function.

Teaching is carried out on a flat surface of uniform color. The background plane
and color model are inferred by looking at parts of the image not contained in
the blue boxes. These are used to keep with high probability only keypoints which
are on the object. During inference (employment), background contributions will
contribute to all examples approximately equally, and any confusion is likely to
be understandable.

\paragraph{Manual Teaching}
Teaching the system about an object involves showing it many different views of
the object in a controlled environment. Once a session is initiated, the following
process should be performed:

\begin{enumerate}
  \item Repeat k times:
  \begin{enumerate}
    \item Adjust the object to expose a novel viewpoint and record pose data.
    \item Remove extraneous objects in the scene to ensure a unique blue box is present.
    \item Collect the view, which saves RGB-D data and the background data.
  \end{enumerate}
\end{enumerate}

Where k might be around 100. Now the collected data can be used to train object class
and pose classifiers.

\paragraph{Semi-Automatic Teaching}

Suppose that we want to gather data while a robot is holding the object. It is possible
to impose geomentric constraints on kept data to ensure that captured keypoints
belong to the object being examined and not to our manipulating robot. This is
called self filtering.


\begin{enumerate}
  \item Repeat k times:
  \begin{enumerate}
    \item The object is placed in a robot's manipulator in a known grasp.
    \item A base pose for the grasp is provided.
    \item The robot collects view from many different precisely known poses, together with models for self and background filtering.
  \end{enumerate}
\end{enumerate}

Where k might be around 3. Eventually we would like to have fully automatic training, where the
robot grabs objects and coordinates base poses itself.

\paragraph{Employing the System}
Object Poses on Table or Floor
Single Target Tracking

\section{Work to be Completed}
Semi-automatic training
Automatic training
More geometric work

working MDP's into the red box exploration and data collection so that we can interact with
large scale planners for tasks.


%\bibliographystyle{siam}
%\bibliography{proposal}

\newpage

\section{Attendance Statement}
  I have been paying attention to AI, Machine Learning, and Computer Vision since 2004. 
Regarding Computer Vision, I saw the progress of Neural Nets in the 90's swept under 
the rug by SIFT, HoG, and SVMs in the mid 2000's, only for neural nets to reclaim the throne in the 2010's.
At one time it was said that AI was "vision hard" and that solving Vision would
effectively solve AI. While that belief sweeps a bit under the rug, it is certainly true
that in the past Computer Vision has been a strong bottleneck in the development of AI and 
Robotics. 
  
  Recent advances in object detection on large data sets suggest that we are ready
to move beyond attacking vision in an isolated setting and begin integrating it in a
larger framework for planning in an interactive environment. I have training in state
of the art Computer Vision techniques and have been tracking the literature for a few
years now. Although I have attended CVPR twice, I have not attended any conferences in
AI or Robotics.  I recently started to focus on Real Time Vision systems, and attending AAAI
will help me rapidly learn about the community and hit the ground running. As my 
concentration shifts and I begin working in AI and Robotics, my assimilation of the literature
will be enchanced by having attended AAAI.
  
  One of my goals is to make a real time vision system with a capacity of 50 objects. It should be
capable of giving accurate 3D pose estimates that enable the objects to be identified and manipulated
by robots. It should be able to learn a new object in less than a minute, and should be simple enough
that a non-expert can easily train the system. Yet it should remain versatile enough that state of
the art detectors running on special hardware can be swapped in by experts.
  
  I have made good progress towards this goal so far. By attending AAAI, I will be able to
better understand the motivations, needs, and desires of AI and Robotics concentrators with regards
to Computer Vision, which will enable me to complete my dissertation in a way that aligns with
these communities' values.

\newpage

\section{Curriculum Vitae}
\textbf{\emph{Education}}\\
 \\
BS in Math, Florida State University, 2003-2006 \\
MA in Math, UC Berkeley, 2006-2008 \\
PhD program in Computer Science at U Chicago, 2010-2011 \\
PhD program in Computer Science at Brown University, 2011-Present \\
 \\ 
\textbf{\emph{Employment}}\\
\\
Developer Support Engineer, Havok, 2008-2009 \\
 \\
\textbf{\emph{Conference Papers}}\\
 \\
S. Naderi Parizi, J. Oberlin, P. Felzenszwalb.\\
\emph{Reconfigurable Models for Scene Recognition.}\\
IEEE Conference on Computer Vision and Pattern Recognition (CVPR), 2012\\
 \\
P. Felzenszwalb, J. Oberlin.  \\
\emph{Multiscale Fields of Patterns.}\\
To Appear, 2014 \\
 \\
\textbf{\emph{Conferences Attended}}\\
 \\
CVPR 2011\\
CVPR 2012 \\
 \\
\textbf{\emph{Teaching Experience}}\\
 \\
Being a TA in the UCB Math Deparment can be close to a full time teaching position, including
quiz design, office hours, and grading in addition to the intense recitation sections (which more closely resembled lectures). \\
\\
Calculus 1 TA\\
UC Berkeley, 2006-2007\\
Two sections in the Fall, three sections in the Spring, 30 students in each section, each section met with me
two or three times a week for a total of about three hours a week.\\
 \\
Linear Algebra and ODE TA\\
UC Berkeley, 2007-2008\\
Same configuration as calculus.\\
 \\
Algorithms Grad TA\\
Brown University, 2012-Present\\
\\
I am currently the Grad TA for the Algorithms class at Brown for the fourth semester. During this time the class has been
between 45 and 100 students each iteration.  I have been responsible for administering oral exams, holding office hours,
lots of grading (most of our problems are proof based), and managing a group of at least 6 Undergraduate TA's each semester.\\
 \\
\textbf{\emph{Departmental Service}}\\
 \\
Graduate Student Orientation Leader\\
Brown CS Department, Fall 2012\\
 \\
Computer Vision Reading Group Coordinator\\
Brown CS Department, 2012\\
 \\
Department Tea Organizer\\
Brown CS Department, 2012-Present\\
 \\
\textbf{\emph{Misc Research}}\\
 \\
Master's Thesis in Mathematics\\
UC Berkeley, Written 2006-2008\\
 \\
Research Experience for Undergraduates in Mathematics\\
Oregon State University, Summer 2005\\
 \\
Undergraduate Research Program in Physics\\
Florida State University, Summer 2004\\
 \\
Research Assistant in Molecular Biology\\
Florida State University, Summer 2003\\
 \\
\textbf{\emph{Hobbies}}\\
 \\
Gardening\\
Ballroom Dance\\
Martial Arts\\
Blacksmithing\\
3D Printing\\
 \\
\newpage

\section{Letter From Supervisor}

\newpage


\end{document}

