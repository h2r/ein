\documentclass[conference]{IEEEtran}
\usepackage{times}
\usepackage{latexsym,amssymb,amsmath} % for \Box, \mathbb, split, etc.
% \usepackage[]{showkeys} % shows label names
\usepackage{cite} % sorts citation numbers appropriately
\usepackage{path}
\usepackage{url}
\usepackage{verbatim}
\usepackage[pdftex]{graphicx}
\usepackage{color}
\usepackage[numbers]{natbib}


\pdfinfo{
   /Author (Homer Simpson)
   /Title  (Robots: Our new overlords)
   /CreationDate (D:20101201120000)
   /Subject (Robots)
   /Keywords (Robots;Overlords)
}


\makeatletter
\setlength{\arraycolsep}{2\p@} % make spaces around "=" in eqnarray smaller
\makeatother


% begin of personal macros
\newcommand{\half}{{\textstyle \frac{1}{2}}}
\newcommand{\eps}{\varepsilon}
\newcommand{\myth}{\vartheta}
\newcommand{\myphi}{\varphi}

\newcommand{\IN}{\mathbb{N}}
\newcommand{\IZ}{\mathbb{Z}}
\newcommand{\IQ}{\mathbb{Q}}
\newcommand{\IR}{\mathbb{R}}
\newcommand{\IC}{\mathbb{C}}
\newcommand{\Real}[1]{\mathrm{Re}\left({#1}\right)}
\newcommand{\Imag}[1]{\mathrm{Im}\left({#1}\right)}

\newcommand{\norm}[2]{\|{#1}\|_{{}_{#2}}}
\newcommand{\abs}[1]{\left|{#1}\right|}
\newcommand{\ip}[2]{\left\langle {#1}, {#2} \right\rangle}
\newcommand{\der}[2]{\frac{\partial {#1}}{\partial {#2}}}
\newcommand{\dder}[2]{\frac{\partial^2 {#1}}{\partial {#2}^2}}

\newcommand{\nn}{\mathbf{n}}
\newcommand{\xx}{\mathbf{x}}
\newcommand{\uu}{\mathbf{u}}

\newcommand{\junk}[1]{{}}

% set two lengths for the includegraphics commands used to import the plots:
\newlength{\fwtwo} \setlength{\fwtwo}{0.45\textwidth}


\renewcommand{\labelitemi}{}
\renewcommand{\labelitemii}{}
\renewcommand{\labelitemiii}{}


% end of personal macros
% \input{inputFile.tex}


\begin{document}
\DeclareGraphicsExtensions{.jpg}


\title{The Eyes of Iorek Byrnison}
\author{Author Names Omitted for Anonymous Review. Paper-ID [add your ID here]}

\maketitle


\begin{abstract}
Robots need to pick stuff up.
%
We feel that perception and planning will both benefit from eachother's feedback.
%
So we contribute a \emph{framework} which includes the structure of
a \emph{memory network} and the \emph{algorithms} (methods) that we use
to maintain the network.
%
Perception can structure data, structured data facilitates planning, planning allows more principled perception.
%
We form a closed loop through perception and planning, allowing them to interact by means of structured data.
%
Our system is supported by online, human-in-the-loop algorithms. Non-technical human participants can easily
teach and collaborate with the system.
\end{abstract}


\section{Problem Setting}

\section{Some Standard Tools}
Here we review the standard AI, vision, and ML algorithms we employ.


\section{Structured Memory Facilitates Planning}
Here we symbolically describe our memory network,
which is the first contribution of the paper and
the first half of our framework.

Working Memory
  What object tokens exist in the world? 
  What are their categories and their names? 
    (names map particular object instances I see now to instances that persist over
     time, that is, for which I have a history)
  Where are they in the scene?
  
Long Term Memory
  Intrinsic (Autobiographical) Episodic Memory
    This is the sensory data that we are collecting.
    This information cannot be provided by external
    sources, it comes from the perceptual capabilities of the system.
    Traditionally this would be thought of as Training Data X.
  Extrinsic Episodic Memory
    Labels and names for categories and peristant instances.
    This information can be provided by another source and will
    eventually be provided by the entity itself.
    Traditionally this would be thought of as Training Labels Y.
  Implicit Memory
    These are the models, such as SVM or kNN, that are involved in
    mapping intrinsic memory to the extrinsic memory.
  Semantic Memory
    Facts about stuff that can be used to reason. Affordances
  Procedural Memory
    Grasps and collision free planning are procedural memory.

Short Term Persistent Memory
  Structurally identical to long term memory but restricted to the current context.
  This is much like a cache for long term memory. Short term memories are incorporated 
  into long term memory either online through the day or offline during a sleep phase.
    


\section{Online Learning With Humans in the Loop}
Here we describe the algorithms which we use to build and grow the memory network,
which are the second contribution and second half of our framework.
Note that performing inference with the framework is actually the same as growing the
memory, since the results of inference are recorded in working and short term memory
directly and strongly contribute to long term memory.



\section{Evaluation}
Our system can be evaluated in two important ways. Firstly, how effective is 
the system at facilitating other actions? Secondly, how accessible to users is the system?

Because this system is designed as a component in a complex system, raw recognition rates
and pose estimation accuracy would not be very informative even if it were not impractical to
obtain them.  Besides, we are not trying to extend the state-of-the-art on those tasks. Rather,
we are trying to provide an interactive framework which will raise the maximum automated vision
available to the average user.

We therefore establish a baseline for performance by training the system in a representative domain
specific setting, which tells us how well it can perform on laboratory objects when trained by an 
expert. This represents the best that the system could be expected to perform.

We then go on to compare the performance of the system when trained to various degrees by naive
and technical non-expert users.

\subsection{Baseline Domain Specific Pick and Place}
We qualitatively evaluate the performance by picking and placing
a respresentative set of objects and reporting the final success rate for each 
object separately.

We report the performance of the system as a function of user interactions.

We report the performance of the system as a function of program lifetime.

Our representative set consists of a block, a spoon,
a bowl, a diaper, and a sippy cup. A \emph{single cut video} showing multiple grasps
of all objects is available here.

\subsection{Usability Experiments}
We repeat the DSPP experiments with naive users in two settings. In the first, they train laboratory
objects. In the second, they provide their own objects.


\citep{tellex11}


\section{Extensions}



\bibliographystyle{plainnat}
\bibliography{main,references}


\end{document}








