\documentclass[conference]{IEEEtran}
\usepackage{times}
\usepackage{latexsym,amssymb,amsmath} % for \Box, \mathbb, split, etc.
% \usepackage[]{showkeys} % shows label names
\usepackage{cite} % sorts citation numbers appropriately
\usepackage{path}
\usepackage{url}
\usepackage{verbatim}
\usepackage[pdftex]{graphicx}
\usepackage{color}
\usepackage[numbers]{natbib}


\pdfinfo{
   /Author (Homer Simpson)
   /Title  (Robots: Our new overlords)
   /CreationDate (D:20101201120000)
   /Subject (Robots)
   /Keywords (Robots;Overlords)
}


\makeatletter
\setlength{\arraycolsep}{2\p@} % make spaces around "=" in eqnarray smaller
\makeatother


% begin of personal macros
\newcommand{\half}{{\textstyle \frac{1}{2}}}
\newcommand{\eps}{\varepsilon}
\newcommand{\myth}{\vartheta}
\newcommand{\myphi}{\varphi}

\newcommand{\IN}{\mathbb{N}}
\newcommand{\IZ}{\mathbb{Z}}
\newcommand{\IQ}{\mathbb{Q}}
\newcommand{\IR}{\mathbb{R}}
\newcommand{\IC}{\mathbb{C}}
\newcommand{\Real}[1]{\mathrm{Re}\left({#1}\right)}
\newcommand{\Imag}[1]{\mathrm{Im}\left({#1}\right)}

\newcommand{\norm}[2]{\|{#1}\|_{{}_{#2}}}
\newcommand{\abs}[1]{\left|{#1}\right|}
\newcommand{\ip}[2]{\left\langle {#1}, {#2} \right\rangle}
\newcommand{\der}[2]{\frac{\partial {#1}}{\partial {#2}}}
\newcommand{\dder}[2]{\frac{\partial^2 {#1}}{\partial {#2}^2}}

\newcommand{\nn}{\mathbf{n}}
\newcommand{\xx}{\mathbf{x}}
\newcommand{\uu}{\mathbf{u}}

\newcommand{\junk}[1]{{}}

% set two lengths for the includegraphics commands used to import the plots:
\newlength{\fwtwo} \setlength{\fwtwo}{0.45\textwidth}


\renewcommand{\labelitemi}{}
\renewcommand{\labelitemii}{}
\renewcommand{\labelitemiii}{}


% end of personal macros
% \input{inputFile.tex}


\begin{document}
\DeclareGraphicsExtensions{.jpg}


\title{NODE: A Robotic Process Virtual Machine}
\author{Author Names Omitted for Anonymous Review. Paper-ID [add your ID here]}

\maketitle


\begin{abstract}
Robots need to pick stuff up.
%
Perception can structure data, structured data facilitates planning, planning allows more principled perception.
%
The tasks of robotic perception, planning, and control will all benefit from
a design paradigm which allows them to be jointly programmed.
%
So we contribute an implementation \emph{NODE} which facilitates the design and 
execution of versatile and scalable MDPs structured in what we call a
Von Neumann MDP pattern. 
%
We form a closed loop through perception and planning, allowing them to interact by means of structured data.
%
Our system incorporates online, human-in-the-loop algorithms. Non-technical human participants can easily
teach and collaborate with the system.
\end{abstract}


\section{Introduction}
% First paragraph: What is the problem we are trying to solve? I
% think it's something about enabling a robot to robustly perceive and
% manipulate native objects, so that it can carry out tasks such as
% assisting at childcare, cooking, or in hospitals.
% Second paragraph: Why hasn't previous work addressed this problem?
% I think it's something about the focus on category recognition, not
% doing pose estimation, and training methods that require an expert
% user and an annotated corpus.
% Third paragraph: What do we propose to do to address this problem?
% Fourth paragraph: Our high-level technical approach
% Fifth paragraph: How do we know it works? Why is it cool?
% Something about an evaluation and something about a robotic
% demonstration.


There are now many types of robots. A common configuration is one that mimics
the human arm and hand system (AHS).  Most tasks utilizing an AHS focus on controlling the
pose of the end effector, which is determined by 6 degrees of freedom.  Thus a hand should have 
at least 6 degrees of freedom, and typical AH systems have at least 7.

A major problem in robotics research is that, while the basic form of AHS bots has remained
similar over time, the systems architectures for controlling those bots tend to be built 
anew. Therefore, when a lab acquires a new robot, a lot of engineering overhead must be paid
in order to get old procedures running on the new bot.

We dramatically increase the portability of robotic procedures by demonstrating a powerful
abstraction framework (NODE) which depends only at a very low level upon the robotic platform
of execution. Porting code to a new AHS is exactly the act of reconciling a simple programmatic
interface with hardware drivers.

Furthermore, NODE is a framework whose structure jointly facilitates perception of 
and planning in the robot's environment. This allows powerful algorithms to be implemented in a
generic, portable environment.

That is to say, NODE is a platform on which we can execute MDPs and related processes.

A Markov decicion process is essentially a state machine with some stochastic planning (reward)
sugar on top. It is no surprise, then, that we address the construction of MDPs with a mindset similar
to that which we use in the construction of computer programs. Take, for instance, object-oriented MDPs. But there
are some things between the state machine of a computer and an object oriented programming language:
machine language (pushing the state machine closer to being a true Turing machine), the Von Neumann computer 
architecture, and its accompanying execution model.

A key point of this architecture is that RAM holds not only program data but also program instructions.
Modern implementations also include convenient hierachical cache structures to facilitate fast,
"real time" access to many more times data than fits in RAM.

In order to quickly construct MDPs that scale and port like modern software, we provide an
archetype system that is to an MDP as a Von Neumann computer is to a Turing machine.
In particular, the state of the MDP includes a call stack of instructions which the MDP
can modify, as well as data pertaining to various machine learning algorithms which it employs.

Actions in a VNMDP are defined in terms of a basic "machine language".

We used the archetype to construct several MDPs over the Baxter robot. This MDP learns to recognize and manipulate
objects which are presented to the bot. We achieve good recognition rates and can learn parameters 
for complex behaviors such as grasping and throwing. Since its instructions are stored in its main memory, 
it has the additional ability to learn actions by writing its own code.



\section{Related Work}
Here we review the standard AI, vision, and ML algorithms we employ.
We use the language of ensemble learning to speak about obtaining strong solutions to 
general problems from weak solutions to specific cases of those problems. 

we are bootstrapping proverbially. our philosophy espouses solving weak instances of 
problems in constrained spaces and using those solutions to create strong solutions 
in unconstrained spaces.  Thereby, we bootstrap the detection process. (To truly be 
faithful to this approach, we must replace the objectness detector, as it uses an 
externally trained model. Something like a high sigma DoG or gradient magnitude should 
do the trick, and it harkens back to Canny.)  

We are not stacking our classifiers in the strictest of senses, but k-means could be 
considered a subordinate classifier so in that sense we are.  Logistic regression is 
a likely candidate for a fusion method when classifying windows of states, and so then 
we might start to truly stack.

We are using bootstrapping to estimate properties such as table orientation and background 
color. as it turns out, it may be worth investigating bagged nearest neighbors because 
we approximating it by resampling the data.

We make heavy use of cascades.

Our training method can be viewed as boosting-motivated because we detect places where
the classifier fails and train a sub-nearest neighbors classifier in that neighborhood.
Mining hard negatives for kNN is the same as boosting a kNN classifier.

relevant actions are proposed, and activities can then be classified in their 
own arc. there might be an action happening at these objects.

the notion of rapids (knn -> svm -> fit) can be naturally motivated by the 
visual system, as has been noted before with BoW and SBoW 
%(http://www.cs.unc.edu/~lazebnik/publications/pyramid_chapter.pdf).

affordances facilitate cascade arrangement. make sure to talk about viola-jones.

%need to switch to a threadpool model which can then swap resources such as CPUs and GPUs 
%as they become available.

%objectness, gradients, and laplacians. consider running objectness on a contrast normalized 
%image. then, consider using gradient and laplacian information as fine grain evidence for 
%the pose rapid. this approach might allow for a principled solution to occlusion.

don’t forget that k-means here is actually a classifier and so we are technically stacking. 
introducing the background shunt will make that cascade-like. a shunt is a special kind of 
cascade step which produces something like a mask.

one approach to teaching the robot to move is to teach it a form for each task. 
each movement in the form should have parameters corresponding to target points, 
known obstacles, etc.  It should be similar to dealing with character animations in video games. 


\section{NODE Architecture}
%Our system, NODE, can be thought of as an emulator for a Von Neumann computer whose CPU 
%instruction set operates natively on robot pose and planning data. Thus in order to employ our system 
%on a new host robot, one need only robustly implement a few atomic robotic operations. 

NODE is a process virtual machine which implements an MDP with a design patterned
after the Von Neumann computer architecture. 

One reason to choose this pattern is that it facilitates the organization of states
and transitions so that we can build MDPs as easily as we build traditional computer programs.

Another reason to choose the VNMDP design pattern is that the majority of execution platforms
(i.e. computers) are modeled after Von Neumann Machines. By sharing that pattern,
we can scale MDPs by taking advantage of the natural, efficiently engineered architecture
which allows computer programs to operate in real time.

A key property of the Von Neumann computer is that it is a stored instruction computer,
meaning that program instructions are stored in its main memory and can be modified
by the computer.  This pattern yields a direct means for our MDP to learn and describe new
behaviours: its instruction set, a language that it interprets during execution.

The NODE CPU (nCPU) consists of a pushdown automaton which utilizes a call stack and a fixed capacity
bank of registers which hold data retrieved from the nCACHE.
The nCPU has an instruction set, which is the set of atomic operations that we want the MDP to be able
to plan with. The instruction set encodes part of the MDP's transition function, is dependent 
totally upon state, and is invariant to actions taken.
The nCPU instructions consist of atomic motor commands (which are passed to and handled by the nARU),
basic linear algebra operations to manipulate vector and pose data, synchronization commands to keep
subsystems in step and coordinated, sensory monitoring commands (resource control is essential), 
planning and policy execution, and any other basic calculation or manipulation that we want the 
MDP to be able to use in its plans.

The NODE Atomic Robot Unit (nARU) implements a set of atomic robotic operations and can be 
thought of as the “hooks” connecting NODE to its host robot. It is a component of the nCPU.

The NODE Cache (nCACHE) is local memory associated with the nCPU and nCACHE. It is the medium
between the nCPU and the main memory hierarchy.

The NODE RAM (nRAM) consists of object models and procedural data that is relevant to the 
executing program. Its construction and maintenance are described by the NODE Short Term Memory 
Model (nSTMM).

The NODE Disk (nDISK) consists of all previously stored memories which have not yet been 
forgotten. It is described by the NODE Long Term Memory Model (nLTMM) and is structurally 
isomorphic to the nSTMM but procedurally different.

The NODE GPU (nGPU) consists of a processor structure analogous to the nCPU but asynchronous from
it, with instructions suited to the implementation of a set of computer vision algorithms.

%The NODE SPU (nSPU) consists of a processor structure analogous to the nCPU but asynchronous from
%it, with instructions suited to the implementation of a set of speech algorithms.




$********$ diagram of pattern $********$


\section{Memory Model: nCACHE, nRAM and nDISK}
Structured memory facilitates planning and the ability to scale while maintaining
high frequency.

Working Memory (nCACHE)
  What object tokens exist in the world? 
  What are their categories and their names? 
    (names map particular object instances I see now to instances that persist over
     time, that is, for which I have a history)
  Where are they in the scene?

Short Term Persistent Memory (nRAM)
  Structurally identical to long term memory but restricted to the current context.
  This is much like a cache for long term memory. Short term memories are incorporated 
  into long term memory either online through the day or offline during a sleep phase.
  
Long Term Memory (nDISK)
  Intrinsic (Autobiographical) Episodic Memory
    This is the sensory data that we are collecting.
    This information cannot be provided by external
    sources, it comes from the perceptual capabilities of the system.
    Traditionally this would be thought of as Training Data X.
  Extrinsic Episodic Memory
    Labels and names for categories and peristant instances.
    This information can be provided by another source and will
    eventually be provided by the entity itself.
    Traditionally this would be thought of as Training Labels Y.
  Implicit Memory
    These are the models, such as SVM or kNN, that are involved in
    mapping intrinsic memory to the extrinsic memory.
  Semantic Memory
    Facts about stuff that can be used to reason. Affordances
  Procedural Memory
    Grasps and collision free planning are procedural memory.


each degree of cascade is appropriate for particular regimes. you can switch regimes in 
real time to accommodate changing environment complexity or resource availability.

at each application, planning includes operator input / real time parameter locking.

a detector cache hierarchy emerges with policies. this immediately induces a dual data hierarchy.

system resources can be bounded by the number of red boxes, which conveniently bounds 
the number of pose detectors (PDs) necessary to keep around at one time. this is a 
fixed length PD cache. but if PDs act to break a tie, than PD-cache inherits structure from CD-cache

class detectors can be regenerated based on context. there is a CD-cache from red 
boxes, a CD-cache from context, and a CD-cache from “hard negative” classes.

time is segmented into phrases with relative frames (such as boundary frames) saved 
as observations. most observations are incomplete, but we can tell which examples we 
can use for which training tasks.  it is an affordance.


\section{Core Model: nCPU}
We use a pushdown automaton with a call stack (CS), several 
integer state variables ($S_i$), and an end effector pose cache (EEPC).

$********$ A subset of the instruction set $********$

Description of mapping of registers to MDP state variables.

Examples of how instructions change the state, i.e. how they
contribute to the transition function.

In addition to being probabilistically entwined with the external environment, 
actions of the MDP operate on the internal environment by modifying the program
running on the virtual machine. 


\section{Robot Model: nARU}
The nARU translates Atomic Robotic Instructions invoked by the nCPU 
into robot driving commands.


\section{Vision Model: nGPU}
This should probably be thought of as some number of CPU cores working
in conjunction with a GPU.

an appealing approach is to train the SVM only on the examples proposed by the kNN models.

a simpler cascade
sliding window on class
mesh based pose estimator such as ICP

current cascade
objectness
knn-bow class label
knn pose label


proposed 1.0 cascade
objectness
class rapids
knn-bow class candidates
discriminative classifier
pose rapids
knn pose label candidates
discriminative pose estimator

a deep cascade
objectness provides evidence; point cloud and other hooks feed in here
knn-bow class suggestions form blue labels for blue boxes; planning can 
place contextual clues here by providing suggestions
restricted discriminative classifier (svm or other) forms red label for 
red boxes; planning can, for instance, fix a red label 
generative model with volumetric representation classifies pose and accounts 
objectness evidence, depth information and object labels (in the case of parts 
and compound objects, this includes dynamics information like constraints)

break the observed space into a block representation, possibly with octree encoding weighted 
according to scale, and use kNN or hashing to search the space for a similar landscape. then 
use the navigation schemes learned for the closest neighbors. 

then we can begin volumetric modeling of objects. each block has an rgb-a value associated 
with it, alpha representing the frequency with which depth is observed at that point.
we can apply markov methods to regularize these models. then it would be fast to move to 
modeling the faces of the blocks with independent filters, treating the faces as patches, to capture texture.

actually, red boxes are likely to become "target patches" rather than objects themselves.
    


\section{Online Learning With Humans in the Loop}
Here we describe the algorithms which we use to build and grow the memory network,
which are the second contribution and second half of our framework.
Note that performing inference with the framework is actually the same as growing the
memory, since the results of inference are recorded in working and short term memory
directly and strongly contribute to long term memory.

We learn the affordance "can be grasped" through RL or just trial and error.
We can report the success rate over time.

there are certain facets of the world which afford actions that are useful to us during 
training. the table and the pointer afford labeling the objects with orientations. 
tables and grounds afford orienting yourself with the world. you could drop an object to 
find “down”. the image background affords object segmentation.

our teaching framework incorporates signs with affordances to streamline the process 
of data collection.

we teach the system by finding its weaknesses and training it to overcome these 
weaknesses. we provide a framework for structuring that training in an affordable, 
scalable way.  we structure the training, the training structures the models.

“I see you’re having trouble picking up gyroBowls. Let’s think about it differently 
(initialize generic pose model) and practice gazing a bit (train the pose model).”

Also, grounding ambiguous object examples to the right class, retraining the models online.

\subsection{Learning Parameters for Fixed MDP Programs}


\subsection{Learning New MDP Programs}


\section{Utility Evaluation}
Our system can be evaluated in two important ways. Firstly, how effective is 
the system at facilitating other actions? Secondly, how accessible to users is the system?

Because this system is designed as a component in a complex system, raw recognition rates
and pose estimation accuracy would not be very informative even if it were not impractical to
obtain them.  Besides, we are not trying to extend the state-of-the-art on those tasks. Rather,
we are trying to provide an interactive framework which will raise the maximum automated vision
available to the average user.

We therefore establish a baseline for performance by training the system in a representative domain
specific setting, which tells us how well it can perform on laboratory objects when trained by an 
expert. This represents the best that the system could be expected to perform.

We then go on to compare the performance of the system when trained to various degrees by naive
and technical non-expert users.

\subsection{Baseline Domain Specific Pick and Place}
We qualitatively evaluate the performance by picking and placing
a respresentative set of objects and reporting the final success rate for each 
object separately.

We report the performance of the system as a function of user interactions.

We report the performance of the system as a function of program lifetime.

Our representative set consists of a block, a spoon,
a bowl, a diaper, and a sippy cup. A \emph{single cut video} showing multiple grasps
of all objects is available here.

\subsection{Usability Experiments}
We repeat the DSPP experiments with naive users in two settings. In the first, they train laboratory
objects. In the second, they provide their own objects.

\section{Ludi Metric}
Training and evaluating with games.

\subsection{Competitive Pick and Place}
Take turns trying to move the object to your opponent's goal while they try to get in your way.

\subsection{Basketball}
Toss your projectiles through your opponent's hoop while guarding your own.


\citep{tellex11}


\section{Extensions}
Right now, NODE runs on Baxter. We will port NODE to PR2 and other AH systems.



\bibliographystyle{plainnat}
\bibliography{main,references}


\end{document}








