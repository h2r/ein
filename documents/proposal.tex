\documentclass[12pt]{article}
\usepackage{latexsym,amssymb,amsmath} % for \Box, \mathbb, split, etc.
% \usepackage[]{showkeys} % shows label names
\usepackage{cite} % sorts citation numbers appropriately
\usepackage{path}
\usepackage{url}
\usepackage{verbatim}
\usepackage[pdftex]{graphicx}
\usepackage{color}

% horizontal margins: 1.0 + 6.5 + 1.0 = 8.5
\setlength{\oddsidemargin}{0.0in}
\setlength{\textwidth}{6.5in}
% vertical margins: 1.0 + 9.0 + 1.0 = 11.0
\setlength{\topmargin}{0.0in}
\setlength{\headheight}{12pt}
\setlength{\headsep}{13pt}
\setlength{\textheight}{625pt}
\setlength{\footskip}{24pt}

\renewcommand{\textfraction}{0.10}
\renewcommand{\topfraction}{0.85}
\renewcommand{\bottomfraction}{0.85}
\renewcommand{\floatpagefraction}{0.90}

\makeatletter
\setlength{\arraycolsep}{2\p@} % make spaces around "=" in eqnarray smaller
\makeatother

% change equation, table, figure numbers to be counted inside a section:
\numberwithin{equation}{section}
\numberwithin{table}{section}
\numberwithin{figure}{section}

% begin of personal macros
\newcommand{\half}{{\textstyle \frac{1}{2}}}
\newcommand{\eps}{\varepsilon}
\newcommand{\myth}{\vartheta}
\newcommand{\myphi}{\varphi}

\newcommand{\IN}{\mathbb{N}}
\newcommand{\IZ}{\mathbb{Z}}
\newcommand{\IQ}{\mathbb{Q}}
\newcommand{\IR}{\mathbb{R}}
\newcommand{\IC}{\mathbb{C}}
\newcommand{\Real}[1]{\mathrm{Re}\left({#1}\right)}
\newcommand{\Imag}[1]{\mathrm{Im}\left({#1}\right)}

\newcommand{\norm}[2]{\|{#1}\|_{{}_{#2}}}
\newcommand{\abs}[1]{\left|{#1}\right|}
\newcommand{\ip}[2]{\left\langle {#1}, {#2} \right\rangle}
\newcommand{\der}[2]{\frac{\partial {#1}}{\partial {#2}}}
\newcommand{\dder}[2]{\frac{\partial^2 {#1}}{\partial {#2}^2}}

\newcommand{\nn}{\mathbf{n}}
\newcommand{\xx}{\mathbf{x}}
\newcommand{\uu}{\mathbf{u}}

\newcommand{\junk}[1]{{}}

% set two lengths for the includegraphics commands used to import the plots:
\newlength{\fwtwo} \setlength{\fwtwo}{0.45\textwidth}


\renewcommand{\labelitemi}{}
\renewcommand{\labelitemii}{}
\renewcommand{\labelitemiii}{}


% end of personal macros
% \input{inputFile.tex}


\begin{document}
\DeclareGraphicsExtensions{.jpg}



\begin{center}
\textbf{\Large }The Eyes of Iorek Byrnison\\[6pt] 
%\textbf{\Large } \\[6pt]
%\textbf{\Large } \\[6pt]
John Oberlin\\
Brown University, 2014\\
\end{center}



\begin{abstract}
State of the art techniques in object detection and pose estimation
are powerful and general but usually run at a rate less than 1 Hz. This makes
it difficult to employ such techniques in real-time human-computer interaction.
This document outlines a simple, robust framework for object detection which 
trades a large memory overhead for improvements in latency and total throughput 
of detections. Included is a workflow for that framework which makes training
and calibration first intuitive and then automatic.
\end{abstract}



\section{Introduction}
%\cite{FooBar}
Our overall pipeline can be described as:

\begin{enumerate}
  \item Collect RGB-D data for objects.
  \item Train BoW model and kNN classifiers for objects.
  \item Use the classifiers and additional logic to provide 3D detections pose estimates of objects.
\end{enumerate}

\section{Basic Techniques}
\paragraph{Objectness} try clustering the proposals for blueboxes.
\paragraph{Fast Keypoints}
\paragraph{SIFT descriptors}
\paragraph{KMeans}
\paragraph{BoW}
\paragraph{Color Histogram}
\paragraph{Depth Histogram}yet to implement
\paragraph{kNN}

\section{Detector Structure}
\paragraph{Green Boxes}
\paragraph{Blue Boxes}
\paragraph{Red Boxes}

\section{Teaching the System}
This is teaching not training because it is an interaction used to collect
data not a script used to optimize a function.

\subsection{Background Filtering}
Teaching is carried out on a flat surface of uniform color. The background plane
and color model are inferred by looking at parts of the image not contained in
the blue boxes. These are used to keep with high probability only keypoints which
are on the object. During inference (employment), background contributions will
contribute to all examples approximately equally, and any confusion is likely to
be understandable.

\subsection{Manual Teaching}
Teaching the system about an object involves showing it many different views of
the object in a controlled environment. Once a session is initiated, the following
process should be performed:

\begin{enumerate}
  \item Repeat k times:
  \begin{enumerate}
    \item Adjust the object to expose a novel viewpoint and record pose data.
    \item Remove extraneous objects in the scene to ensure a unique blue box is present.
    \item Collect the view, which saves RGB-D data and the background data.
  \end{enumerate}
\end{enumerate}

Where k might be around 100. Now the collected data can be used to train object class
and pose classifiers.

\subsection{Self Filtering}
Suppose that we want to gather data while a robot is holding the object. It is possible
to impose geomentric constraints on kept data to ensure that captured keypoints
belong to the object being examined and not to our manipulating robot. This is
called self filtering.

\subsection{Semi-Automatic Teaching}

\begin{enumerate}
  \item Repeat k times:
  \begin{enumerate}
    \item The object is placed in a robot's manipulator in a known grasp.
    \item A base pose for the grasp is provided.
    \item The robot collects view from many different precisely known poses, together with models for self and background filtering.
  \end{enumerate}
\end{enumerate}

Where k might be around 3.

\subsection{Automatic Teaching}
\begin{enumerate}
  \item Place an object in front of the robot.
  \item Robot repeats k times:
  \begin{enumerate}
    \item Infer a stateless grasp on the object.
    \item if a pose model exists, infer the base pose for the grasp. else, start a new pose model.
    \item Perform semi-automatic teaching with the object.
    \item Replace the object.
  \end{enumerate}
\end{enumerate}

Where k might be around 3.

\section{Employing the System}
\paragraph{Object Poses on Table or Floor}
\paragraph{Single Target Tracking}

\section{Yet To Be Implemented}

\begin{itemize}
  \item Background Filtering
  \item Depth Models
  \item Train feature weights to minimize the leave-one-out error rate on training data.
\end{itemize}

\bibliographystyle{siam}
\bibliography{proposal}

\end{document}

