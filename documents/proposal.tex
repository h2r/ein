\documentclass[12pt]{article}
\usepackage{latexsym,amssymb,amsmath} % for \Box, \mathbb, split, etc.
% \usepackage[]{showkeys} % shows label names
\usepackage{cite} % sorts citation numbers appropriately
\usepackage{path}
\usepackage{url}
\usepackage{verbatim}
\usepackage[pdftex]{graphicx}
\usepackage{color}

\usepackage{multicol}

% horizontal margins: 1.0 + 6.5 + 1.0 = 8.5
\setlength{\oddsidemargin}{0.0in}
\setlength{\textwidth}{6.5in}
% vertical margins: 1.0 + 9.0 + 1.0 = 11.0
\setlength{\topmargin}{0.0in}
\setlength{\headheight}{12pt}
\setlength{\headsep}{13pt}
\setlength{\textheight}{625pt}
\setlength{\footskip}{24pt}

\renewcommand{\textfraction}{0.10}
\renewcommand{\topfraction}{0.85}
\renewcommand{\bottomfraction}{0.85}
\renewcommand{\floatpagefraction}{0.90}

\makeatletter
\setlength{\arraycolsep}{2\p@} % make spaces around "=" in eqnarray smaller
\makeatother

% change equation, table, figure numbers to be counted inside a section:
\numberwithin{equation}{section}
\numberwithin{table}{section}
\numberwithin{figure}{section}

% begin of personal macros
\newcommand{\half}{{\textstyle \frac{1}{2}}}
\newcommand{\eps}{\varepsilon}
\newcommand{\myth}{\vartheta}
\newcommand{\myphi}{\varphi}

\newcommand{\IN}{\mathbb{N}}
\newcommand{\IZ}{\mathbb{Z}}
\newcommand{\IQ}{\mathbb{Q}}
\newcommand{\IR}{\mathbb{R}}
\newcommand{\IC}{\mathbb{C}}
\newcommand{\Real}[1]{\mathrm{Re}\left({#1}\right)}
\newcommand{\Imag}[1]{\mathrm{Im}\left({#1}\right)}

\newcommand{\norm}[2]{\|{#1}\|_{{}_{#2}}}
\newcommand{\abs}[1]{\left|{#1}\right|}
\newcommand{\ip}[2]{\left\langle {#1}, {#2} \right\rangle}
\newcommand{\der}[2]{\frac{\partial {#1}}{\partial {#2}}}
\newcommand{\dder}[2]{\frac{\partial^2 {#1}}{\partial {#2}^2}}

\newcommand{\nn}{\mathbf{n}}
\newcommand{\xx}{\mathbf{x}}
\newcommand{\uu}{\mathbf{u}}

\newcommand{\junk}[1]{{}}

% set two lengths for the includegraphics commands used to import the plots:
\newlength{\fwtwo} \setlength{\fwtwo}{0.45\textwidth}


\renewcommand{\labelitemi}{}
\renewcommand{\labelitemii}{}
\renewcommand{\labelitemiii}{}


% end of personal macros
% \input{inputFile.tex}


\begin{document}
\DeclareGraphicsExtensions{.jpg}



\begin{center}
\textbf{\Large AAAI Fellowship Application}\\[12pt] 
%\textbf{\Large } \\[6pt]
%\textbf{\Large } \\[6pt]
John Oberlin\\
Brown University, 2014\\
\end{center}

\section{Introduction}
State of the art techniques in object detection and pose estimation
are powerful and general but usually run at a rate much less than 1 Hz and require
time and expertise to build, maintain, and operate. The high demands of modern systems
can make it difficult to employ such techniques in real-time human-computer interaction.

During the completion of my PhD thesis at Brown, I will make it possible for a robot
to autonomously scan 10 novel objects in order to construct robust models for detection,
pose estimation, grasping, and manipulation of those objects during collaborations with
a human operator. A high level description of the workflow that I use at the moment is:

\paragraph{The Workflow}
\begin{enumerate}
  \item Collect RGB-D data for objects from many different perspectives.
  \item Train BoW model and kNN classifiers for objects.
  \item Use the classifiers and additional logic to provide 3D detections and pose estimates of objects.
\end{enumerate}

It is possible to execute this workflow for 5 objects in 15 minutes, producing models which are
viable for tabletop object detection. Generating models robust enough for general detection might
take three times as long.

Popular techniques for real-time detection use modified DPMs \cite{kostas1} \cite{forsyth1},
and sometimes exploit different channels of data \cite{dietr1} \cite{sliding1}.  
These approaches have seen success in their target domains, but are too technical for general
application and need to be integrated with interactive systems. I want to create a Computer 
Vision system that is simple, reliable, and easy to use with ROS (Robot Operating System).

\section{Coordinating AI with Computer Vision and Teaching the System}
My current framework for detection uses a box metaphor to talk about space in a way that 
is amenable to planning and reasoning. The system uses RGB-D video from a Kinect as input.
Green Boxes denote areas in the image that probably belong to an object. Each Blue Box denotes
a cluster of green boxes that probably belong to the same object. Whereas Blue Boxes exist only for
one frame of video input, Red Boxes denote more precise clusters of Green Boxes which are tracked over time 
and account for occlusion.

Right now, detections and pose estimates are provided for Blue and Red Boxes. I want to allow the system
to provide information about tokenization, affordances, and other properties that are useful for applying
MDPs\cite{MDP}, OO-MDPs\cite{OOMDP}, and POMDPs\cite{POMDP} to the real world.

The Red Box optimization process can be framed as an MDP. I would like to use more sophisticated planning
that incorporates feedback with the robot's movement planner in order to collect additional, specific data 
as it is needed.

Finally, consider the following automatic object registration technique.  
The logic in Step 3 of The Workflow might utilize Amazon Mechanical Turk 
or an existing, more general classifier to automatically label and retrieve meta information for learned 
objects, meaning that the operator only needs to annotate data if the automatic registration step fails.

\begin{figure}
  \begin{center}
    \begin{tabular}{l c}
      \includegraphics[width=200px, height=150px]{robo2.png} &
      \includegraphics[width=200px, height=150px]{screen2.png} \\
    \end{tabular}
  \end{center}
  \caption{Teaching a robot to identify and manipulate objects can be as easy as bringing home
	    a bag of groceries.}
\end{figure}

%\paragraph{Semi-Automatic Training}
%\begin{enumerate}
  %\item A human operator places the object in a robot's manipulator.
  %\item The operator provides a base pose for the given grasp.
  %\item The robot collects views of many different precisely known poses, together with models for self and background filtering.
%\end{enumerate}

\section{Current Work, Future Progress and Broader Impact}
At the moment, a human operator trains the system without a robot by manually adjusting the pose of the object, 
achieving robust detection and pose estimation under good conditions. Eventually we would like to have 
fully automatic training, where the robot grabs objects and coordinates base poses itself.

The system we have outlined abstracts the detector from the classifier, taking the notion of Objectness and
extending it. As more sophisticated Computer Vision algorithms become real-time tractable or new data becomes
available, they might be smoothly incorporated into the system even as it is running. 
Automatic model validation can ensure that a new model outperforms the current model before the system switches 
it out with no interruption of service. This is important if robots are
to become effortless for non-technical operators.


\newpage

\bibliographystyle{siam}
\bibliography{proposal}

\newpage

\section{Attendance Statement}

  I have been paying attention to AI, Machine Learning, and Computer Vision since 2004. 
Regarding Computer Vision, I saw the progress of Neural Nets in the 90's swept under 
the rug by SIFT, HoG, and SVMs in the mid 2000's, only for neural nets to reclaim the throne in the 2010's.
At one time it was said that AI was "vision hard" and that solving Vision would
effectively solve AI. While that belief sweeps a bit under the rug, it is certainly true
that in the past Computer Vision has been a strong bottleneck in the development of AI and 
Robotics. 
  
  Recent advances in object detection on large data sets suggest that we are ready
to move beyond attacking vision in an isolated setting and begin integrating it in a
larger framework for planning in an interactive environment. I have training in state
of the art Computer Vision techniques and have been tracking the literature for a few
years now. 

  Work in Computer Vision has largely centered around hyperspecific problems concerning
single images. This focus has resulted in significant progress on such tasks, but the challenges of 
engineering real time systems has so far prevented interesting real world applications
or substantial spread of techniques to other communities. 

Two things have resulted from this dam in the flow of information. 
First, there are now many techniques in Computer Vision which are suited
to fast and effective partial solutions of problems such as object detection
but have been ignored in favor of much slower but slightly more effective state of the art
approaches.  Second, such techniques may be combined with the information available from
extra sensors and the ability of real time systems to capture additional images of the same scene
to form full solutions to multiple related problems (such as detection, segmentation, and pose estimation). 
 
I have seen already that combining elements of probabilistic planning and reasoning can boost
the performance of traditional Computer Vision and Machine Learning algorithms.  I want to
continue this approach to research, and attending AAAI will help me to fill in gaps in my
background that are necessary to carve out a career in multi-disciplinary AI.  
Planning and Markov Decision Processes are key elements that I want to master. Additionally,
I would like to explore topics such as Interactive Entertainment, Scheduling, Knowledge
Representation and Reasoning, and Reasoning Under Uncertainty in order to further my
understanding of Human-AI Interactions.

Finally, by attending AAAI I will be able to better understand the motivations, needs, and desires 
of AI and Robotics concentrators with regards to Computer Vision, which will enable me 
to complete my dissertation in a way that aligns with these communities' values.

\newpage

\section{Curriculum Vitae}
\textbf{\emph{Education}}\\
 \\
BS in Math, Florida State University, 2003-2006 \\
MA in Math, UC Berkeley, 2006-2008 \\
PhD program in Computer Science at U Chicago, 2010-2011 \\
PhD program in Computer Science at Brown University, 2011-Present \\
 \\ 
\textbf{\emph{Employment}}\\
\\
Developer Support Engineer, Havok, 2008-2009 \\
 \\
\textbf{\emph{Conference Papers}}\\
 \\
S. Naderi Parizi, J. Oberlin, P. Felzenszwalb.\\
\emph{Reconfigurable Models for Scene Recognition.}\\
IEEE Conference on Computer Vision and Pattern Recognition (CVPR), 2012\\
 \\
P. Felzenszwalb, J. Oberlin.  \\
\emph{Multiscale Fields of Patterns.}\\
To Appear, 2014 \\
 \\
\textbf{\emph{Conferences Attended}}\\
 \\
CVPR 2011\\
CVPR 2012 \\
 \\
\textbf{\emph{Teaching Experience}}\\
 \\
Being a TA in the UCB Math Deparment can be close to a full time teaching position, including
quiz design, office hours, and grading in addition to the intense recitation sections (which more closely resembled lectures). \\
\\
Calculus 1 TA\\
UC Berkeley, 2006-2007\\
Two sections in the Fall, three sections in the Spring, 30 students in each section, each section met with me
two or three times a week for a total of about three hours a week.\\
 \\
Linear Algebra and ODE TA\\
UC Berkeley, 2007-2008\\
Same configuration as calculus.\\
 \\
Algorithms Grad TA\\
Brown University, 2012-Present\\
\\
I am currently the Grad TA for the Algorithms class at Brown for the fourth semester. During this time the class has been
between 45 and 100 students each iteration.  I have been responsible for administering oral exams, holding office hours,
lots of grading (most of our problems are proof based), and managing a group of at least 6 Undergraduate TA's each semester.\\
 \\
\textbf{\emph{Departmental Service}}\\
 \\
Graduate Student Orientation Leader\\
Brown CS Department, Fall 2012\\
 \\
Computer Vision Reading Group Coordinator\\
Brown CS Department, 2012\\
 \\
Department Tea Organizer\\
Brown CS Department, 2012-Present\\
 \\
\textbf{\emph{Misc Research}}\\
 \\
Master's Thesis in Mathematics\\
UC Berkeley, Written 2006-2008\\
 \\
Research Experience for Undergraduates in Mathematics\\
Oregon State University, Summer 2005\\
 \\
Undergraduate Research Program in Physics\\
Florida State University, Summer 2004\\
 \\
Research Assistant in Molecular Biology\\
Florida State University, Summer 2003\\
 \\
\textbf{\emph{Hobbies}}\\
 \\
Gardening\\
Ballroom Dance\\
Martial Arts\\
Blacksmithing\\
3D Printing\\
 \\
\newpage

\section{Letter From Supervisor}

\newpage


\end{document}

